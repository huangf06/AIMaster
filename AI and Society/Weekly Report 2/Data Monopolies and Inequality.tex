\documentclass[11pt]{article}
\usepackage{url}
\usepackage{hyperref}
\hypersetup{
    breaklinks=true,
    colorlinks=true,
    linkcolor=blue,
    filecolor=magenta,      
    urlcolor=cyan,
}

\title{AI and Society: Data Monopolies and Inequality}

\begin{document}
\maketitle

\section{Introduction}

The rapid advancement of artificial intelligence and automation is reshaping our lives profoundly. This report examines two issues that are increasingly urgent to address: data monopolies and inequality. 

For the first one, as tech giants have become more and more visible in the way they affect people's lives, are concerns about the power of tech companies a true reflection of reality, or are they simply the rhetoric of over-worried people? If this threat to power is real, how did it come about, and are today's regulatory efforts moving in the right direction, or are they intentionally or unintentionally ignoring the crux of the problem? 

For the second one, as there is a growing concern on the impact that AI will have on the world of the future, will it further exacerbate inequality in the world of the future, or can it be effectively regulated and used so that the power of technology truly enhances the well-being of all of humankind? What do we need to do about it?

\section{Topic 1: Data Monopolies}

\subsection{Press Perspectives}

The issue of data monopoly is increasingly surfacing, becoming more and more hotly debated, and is becoming an undeniable problem that cannot be ignored. 
Just as the title of an article states, "monopoly power is the elephant in the room in the AI debate"\cite{vonThun2023}. 

\subsubsection{Where exactly does the power to monopolize data come from? }

High-quality data is what gives these tech giants a key strategic advantage. "it is bruisingly expensive to train a new AI system from scratch, and only a few companies — primarily the world’s tech giants — have access to enough data to do it well"\cite{Chatterjee2023}. When it comes to AI, better data always wins. Another  article from Forbes also points out that Amazon and Apple built their monopolies by having detailed access to their users’ data, which in turn allowed them to build the best AI recommendation systems for their respective ecosystems. In addition, Facebook and Google built their monopolies on the network graphs between their users and their preferred content. \cite{Stacey2020}


\subsubsection{What's the problem with having only a handful of players at the top?}

"If artificial intelligence lives up to its promise and becomes the lifeblood of every sector of the economy, we can expect a future of economic concentration and corporate political power that dwarfs anything that came before", an article from TheAsset points out that,  "Big Tech firms now resemble banks in their influence across the economy – but at a supercharged level. Through their access to data, they know more about consumer and business behaviour, and exert more control over it, than banks ever did. They supply vital inputs to businesses across the economy, as well as products and services to almost all consumers. No bank has ever had such reach."\cite{Posner2024}

\subsubsection{What can policymakers do in the face of such problems?}

The previous Politico article \cite{Chatterjee2023} also suggests, one potential government solution is to create a public resource where researchers can learn about the emerging capabilities and limitations of the technology. In AI, this looks like building a publicly funded large-scale language model with accompanying datasets and computational resources for researchers to use.

\subsection{Scientific Perspectives}

Catherine Mulligan argues in her paper\cite{mulligan2023datalism} that, the increasing use of data in various parts of the economic and social systems is creating a new form of monopoly: data monopolies. The Datalists are challenging the existing definitions used within Monopoly Capital Theory (MCT). They are pursuing monopolistic control over data to feed their productive processes, increasingly controlled by algorithms and Artificial Intelligence (AI). These productive processes use information about humans and the creative outputs of humans as the inputs but do not classify those humans as employees, so they are not paid or credited for their labour.

Mason Marks in his paper\cite{marks2021biosupremacy} introduces the concept of biopower. Technology companies derive their power from the data they collect and the intelligence gained through surveillance is used to manipulate people's behavior. These two networks of surveillance and control together form a global digital panopticon. The article suggests that antitrust regulators should broaden their conception of consumer welfare, revive conglomerate merger control, prohibit the use of dark patterns, and mandate data silos.

Daniel Macintosh argued that, what allows the big tech companies to monopolize their business is the newtwork effect acting on data in a positive feedback loop. And competition law has been the common approach to solving the problem. However, such regulations have proved mostly ineffective because the data do not fit neatly into traditional economic models. Another traditional alternative is consumer law, but its main focus is on individual privacy, not monopoly power. So, reimagined competition and consumer regulations may work to prevent inflated prices and Draconian privacy policies, they will not address the more pressing problems of Big Tech monopolies on data.\cite{mcintosh2018we}

\subsection{Comparison}

Overall, the media and academia are basically in agreement about the existence and influence of the data monopoly problem, and academics have given more in-depth studies and explanations about the formation of this power. As to how to solve this problem, academics suggest that current regulatory efforts may not be truly effective in checking monopoly power.

\section{Topic 2: Inequality}

\subsection{Media Views}

For Brynjolfsson, an economist, simple automation, while producing value, can also be a path to greater inequality of income and wealth. The emphasis on automation rather than augmentation is, he argues in the essay, the “single biggest explanation” for the rise of billionaires at a time when average real wages for many Americans have fallen. \cite{Rotman2022} 

Joseph E, the Nobel Prize-winning economist, also warned that unfettered capitalism, unfettered innovation, does not lead to the general well-being of our society. Artificial intelligence has already eliminated thousands of jobs and could eventually lead to the automation of hundreds of millions. Left unchecked, this labor disruption could further concentrate wealth in the hands of corporations, leaving workers with less power than ever before. We have created a system where workers have little bargaining power. In such a world, AI could be an ally of employers, further eroding workers' bargaining power and further increasing inequality.\cite{Sophie2013}

Economists attending the Davos Forum, meanwhile, expect AI to affect the world economy unequally. While 94\% of the economists surveyed expect AI to radically increase productivity in high-income economies over the next five years, only 53\% predict a similar impact on low-income economies. And 87\% of economists predicted that the impact of AI is expected to exacerbate volatility in the global economy as geopolitical developments unfold.57\% also predicted that these conditions will exacerbate inequality and widen the North-South divide over the next three years.\cite{AlJazeera2024}

\subsection{Scientific Views}

A study \cite{goyal2020artificial} investigates how technologies such as artificial intelligence (directly or indirectly) affect workers' job positions and work processes. The analysis of the study was based on data on automation and the Gini coefficient. It found that the relationship between AI and income distribution has been considered negative, which is what is observed in this study, and that it directly affects the distribution of income and employment. Due to automation, low and middle skill jobs are declining, unemployment is rising and the income gap between middle and high skill labor is widening further.

In terms of how to try to address inequality, academics have called for the introduction of sociological approaches to the study of AI. In an article\cite{joyce2021toward} by Kelly Joyce, it is noted that AI practitioner-led efforts typically fail to recognize the full complexity of social life, missing the opportunity to fully consider how power dynamics, systemic discrimination and social inequality contribute to the meaning of social categories.
And sociological research has shown that biases are not free-floating within individuals, but are rooted in entrenched social institutions. In other words, transparency and fairness are not easily achieved through better training of AI practitioners.


Laura Sartori and Andreas Theodorou emphasizes in their article\cite{sartori2022sociotechnical} that if we consider AI as a sociotechnical system, we are to include all participants in the process of construction in a co-creation approach. Research could shed light on the legitimization mechanisms underlying the relationship between social and artificial agents. The relevance of narratives in shaping current realities is a strong call for citizens-with their perceptions and beliefs -to sit at the table for the future of AI.


Mike Zajko's article\cite{zajko2022artificial} indicates social inequality is real, and that this inequality is always present in the data used to train algorithms through machine learning, as well as in the output of these systems. And any attempt to find a solution is inherently political; avoiding the problem or treating it as outside the scope of data science amounts to a conservative orientation. And the loss of human agency is a recurring concern in discussions about automation and AI. structured inequality has long deprived marginalized groups of certain forms of agency. And sociologists can help assert agency over new technologies through three kinds of actions: (1) critique and the politics of refusal; (2) fighting inequality through technology; and (3) governance of algorithms.

\subsection{Comparison}

There is not much controversy in academia and the media about technology exacerbating inequality, and attempts have been made to propose ways to address inequality from a sociological perspective.

\section{Conclusion}

The problem of data monopoly exists, with tech giants monopolizing biopower by collecting data through surveillance networks and exerting influence by controlling the networks, and with information about human beings and their creative outputs being used without compensation, leading to a new era of monopoly capital. And because data monopolies, unlike past competition for products and services, are difficult to regulate by competition regulations, while consumer laws focus more on privacy and have limited checks and balances on monopoly power.

It is true that inequality has increased with the technological development of artificial intelligence and automation. And addressing these inequalities requires more sociological considerations, and the development of AI needs to be integrated with sociology to fully take into account power dynamics, systemic discrimination and social inequalities. Whereas structural inequality is rooted in the deprivation of certain agency for marginalized groups over time, sociologists can contribute to addressing the loss of artificial agency by critiquing and rejecting the politics, fighting inequality through technology; and governance of algorithms to address the loss of structural inequality.

\begin{thebibliography}{12}

\bibitem{vonThun2023}
Max von Thun.
\textit{Monopoly Power Is the Elephant in the Room in the AI Debate}.
TechPolicy, October 23, 2023.
Available at \url{https://www.techpolicy.press/monopoly-power-is-the-elephant-in-the-room-in-the-ai-debate/}.

\bibitem{Chatterjee2023}
Mohar Chatterjee.
\textit{AI might have already set the stage for the next tech monopoly}.
POLITICO, March 22, 2023.
Available at \url{https://www.politico.com/newsletters/digital-future-daily/2023/03/22/ai-might-have-already-set-the-stage-for-the-next-tech-monopoly-00088382}.

\bibitem{Stacey2020}
Ed Stacey.
\textit{Emerging AI Will Drive The Next Wave Of Big Tech Monopolies}.
Forbes, October 28, 2020.
Available at \url{https://www.forbes.com/sites/edstacey/2020/10/28/emerging-ai-will-drive-the-next-wave-of-big-tech-monopolies/?sh=7aca78525512}.

\bibitem{Posner2024}
Eric Posner.
\textit{AI revolution likely to cement Big Tech monopoly}.
The Asset, January 11, 2024.
Available at \url{https://www.theasset.com/article/50713/ai-revolution-likely-to-cement-big-tech-monopoly}.

\bibitem{mulligan2023datalism}
Catherine E. A. Mulligan and Phil Godsiff.
\textit{Datalism and Data Monopolies in the Era of A.I.: A Research Agenda}.
2023.

\bibitem{marks2021biosupremacy}
Mason Marks.
\textit{Biosupremacy: Big Data, Antitrust, and Monopolistic Power Over Human Behavior}.
UC Davis Law Review, 55, 513, 2021.

\bibitem{mcintosh2018we}
Daniel McIntosh.
\textit{We need to talk about data: how digital monopolies arise and why they have power and influence}.
Journal of Technology Law \& Policy, 23, 185, 2018.

\bibitem{Rotman2022}
David Rotman.
\textit{AI is making inequality worse}.
MIT Technology Review, April 19, 2022.
Available at \url{https://www.technologyreview.com/2022/04/19/1049378/ai-inequality-problem/}.

\bibitem{Sophie2013}
Sophie Bushwick.
\textit{Unregulated AI Will Worsen Inequality, Warns Nobel-winning Economist Joseph Stiglitz}.
Scientific American, August 1, 2023.
Available at \url{https://www.scientificamerican.com/article/unregulated-ai-will-worsen-inequality-warns-nobel-winning-economist-joseph-stiglitz/}.

\bibitem{AlJazeera2024}
Al Jazeera.
\textit{Geopolitics, AI to slow global economy, grow inequality: Davos survey}.
Al Jazeera and New Agencies, January 15, 2024.
Available at \url{https://www.aljazeera.com/news/2024/1/15/global-economy-8}.

\bibitem{goyal2020artificial}
Arjun Goyal and Ranjan Aneja.
\textit{Artificial intelligence and income inequality: Do technological changes and worker's position matter?}.
Journal of Public Affairs, 20(4):e2326, 2020.

\bibitem{joyce2021toward}
Kelly Joyce, Laurel Smith-Doerr, Sharla Alegria, Susan Bell, Taylor Cruz, Steve G Hoffman, Safiya Umoja Noble, and Benjamin Shestakofsky.
\textit{Toward a sociology of artificial intelligence: A call for research on inequalities and structural change}.
Socius, 7, 2378023121999581, 2021.
SAGE Publications Sage CA: Los Angeles, CA.

\bibitem{sartori2022sociotechnical}
Laura Sartori and Andreas Theodorou.
\textit{A sociotechnical perspective for the future of AI: narratives, inequalities, and human control}.
Ethics and Information Technology, 24(1):4, 2022.
Springer.

\bibitem{zajko2022artificial}
Mike Zajko.
\textit{Artificial intelligence, algorithms, and social inequality: Sociological contributions to contemporary debates}.
Sociology Compass, 16(3):e12962, 2022.
Wiley Online Library.


\end{thebibliography}
\end{document}
